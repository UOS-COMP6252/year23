\documentclass{beamer}
\usepackage{tikz}
\usepackage{verbatim}
\usepackage{amsmath}
\usepackage{hyperref}
\usepackage{array,booktabs}
\usepackage{listings}
\usepackage{xcolor}
\usepackage{colortbl}
\usepackage{wrapfig}

\usepackage{mathtools}
\DeclarePairedDelimiter\ceil{\lceil}{\rceil}
\DeclarePairedDelimiter\floor{\lfloor}{\rfloor}


\usepackage[utf8]{inputenc}
\usepackage{algorithm}
\usepackage[noend]{algpseudocode}

\definecolor{pblue}{rgb}{0.13,0.13,1}
\definecolor{pgreen}{rgb}{0,0.5,0}
\definecolor{pred}{rgb}{0.9,0,0}
\definecolor{pgrey}{rgb}{0.46,0.45,0.48}
\definecolor{cadmiumgreen}{rgb}{0.0, 0.42, 0.24}

\usepackage{listings}
\lstset{language=Python,
  showspaces=false,
  showtabs=false,
  breaklines=true,
  showstringspaces=false,
  breakatwhitespace=true,
  commentstyle=\color{pgreen},
  keywordstyle=\color{pblue},
  stringstyle=\color{pred},
  basicstyle=\small\ttfamily,
  moredelim=[il][\textcolor{pgrey}]{$$},
  moredelim=[is][\textcolor{pgrey}]{\%\%}{\%\%}
}

\usetikzlibrary{automata,mindmap,graphs}
\usetikzlibrary{shapes}


\usepackage{hus-beamer} % for inserting HUS logos at the top-right and bottom-left corners, modify hus-beamer.sty for removing or re-positioning the logos
%\usepackage[sfdefault]{roboto} 
\hypersetup{
    unicode=true,          % non-Latin characters in Acrobat’s bookmarks
    pdfencoding=unicode,
    pdftoolbar=true,        % show Acrobat’s toolbar?
    pdfmenubar=true,        % show Acrobat’s menu?
    pdffitwindow=false,     % window fit to page when opened
    pdfstartview={FitH},    % fits the width of the page to the window
    pdftitle={HUS-Beamer Template},    % title
    pdfauthor={Hikmat Farhat},     % author
    pdfsubject={Inductive invariants in a continuous setting},   % subject of the document
    pdfcreator={Hikmat Farhat},   % creator of the document
    pdfproducer={LaTeX}, % producer of the document
    pdfnewwindow=true,      % links in new window
    colorlinks=false,       % false: boxed links; true: colored links
    linkcolor=false,          % color of internal links (change box color with linkbordercolor)
    urlcolor=false}
    
\usepackage[british]{babel}

\mode<presentation>{
	\usefonttheme{professionalfonts} % normal font for math formulas
	% insert section page with title only
	% before each section
	\AtBeginSection[]{
	\begin{frame}%[noframenumbering] % remove this if you do not want to number section page
	\vfill
	\centering
	\begin{beamercolorbox}[sep=8pt,center,shadow=true,rounded=true]{title}
	\usebeamerfont{title}\insertsectionhead\par%
	\end{beamercolorbox}
	\vfill
	\end{frame}
}
}

% Some common packages
\usepackage{amsmath}
\usepackage{amsfonts}
\usepackage{amssymb}

\setbeamertemplate{theorems}[numbered] % numbering theorem environment

%%% New environment

\theoremstyle{plain} % other style are definition and example, you can also define your own style
\newtheorem{proposition}{Proposition}

%%% Numbering examples with a separate counter, comment the below if not use
\usepackage{etoolbox} % for \undef command => re-define environment
\undef{\problem}
\newtheorem{problem}{Problem}
\undef{\example}
\theoremstyle{example} % style of the example environment
\newtheorem{example}{Example}

\newcommand{\deriv}[2]{\ensuremath{\frac{\partial #1}{\partial #2}}}
\newcommand{\INf}{\mathrm{In}_{f}}
\newcommand{\INnf}{\mathrm{In}_{-f}}
\newcommand{\rk}{\mathrm{rk}}
\newcommand{\red}[1]{{\bf \textcolor{red}{#1}}}
\newcommand{\boldf}[1]{{\bf #1}}

\newcommand{\ttt}[1]{\texttt{#1}}
\newcommand{\ttb}[1]{\textcolor{blue}{\texttt{#1}}}
\newcommand{\ttr}[1]{\textcolor{red}{\texttt{#1}}}

% Biblatex in beamer
\usepackage[bibstyle=authoryear, citestyle=authoryear, maxcitenames=2, maxbibnames=100, backend=bibtex]{biblatex}

\AtBeginBibliography{\footnotesize} % Footnotesize for Bibliography entries

\setbeamertemplate{bibliography item}{%
  \ifboolexpr{ test {\ifentrytype{book}} or test {\ifentrytype{mvbook}}
    or test {\ifentrytype{collection}} or test {\ifentrytype{mvcollection}}
    or test {\ifentrytype{reference}} or test {\ifentrytype{mvreference}} }
    {\setbeamertemplate{bibliography item}[book]}
    {\ifentrytype{online}
       {\setbeamertemplate{bibliography item}[online]}
       {\setbeamertemplate{bibliography item}[article]}}%
  \usebeamertemplate{bibliography item}}

\defbibenvironment{bibliography}
  {\list{}
     {\settowidth{\labelwidth}{\usebeamertemplate{bibliography item}}%
      \setlength{\leftmargin}{\labelwidth}%
      \setlength{\labelsep}{\biblabelsep}%
      \addtolength{\leftmargin}{\labelsep}%
      \setlength{\itemsep}{\bibitemsep}%
      \setlength{\parsep}{\bibparsep}}}
  {\endlist}
  {\item}

\addbibresource{refs.bib}

\defbibheading{bibliography}[\refname]{}

% Info
\title{Backpropagation}
%\subtitle{Lesson 2}
\author{COMP6252 (Deep Learning Technologies)}
\institute[ECS, University of Southampton]{ECS, University of Southampton} \date{}

\begin{document}

\begin{frame}[plain,noframenumbering]
\placelogofalse % No logo at the title page
\titlepage
\end{frame}

\placelogotrue % Logo at other pages
\begin{frame}
    \frametitle{Computing the gradient}
\begin{itemize}
    \item An essential part of Deep Learning is computing the gradient of some loss function
    \item In most situations one cannot compute the gradient analytically
    \item PyTorch computes the gradient using three concepts
    \begin{enumerate}
        \item Computational Graph
        \item Maintain a list of the derivatives of \textbf{primitive operations}
        \item Use the chain rule from calculus
    \end{enumerate}
\end{itemize}
    

\end{frame}


\begin{frame}
    \frametitle{Chain rule}
\begin{itemize}
    \item The chain rule allows us to compute the derivative of a composite function. Consider the following example
\end{itemize}
\begin{align*}
    h=g(f(x))
\end{align*}
\begin{itemize}
    \item To  compute $\frac{dh}{dx}$, the chain rule gives
\end{itemize}    
\begin{align*}
    \frac{dh}{dx}=\frac{dh}{dg}\frac{dg}{df}\frac{df}{dx}
\end{align*}
\begin{itemize}
    \item The crucial point in the above is that if we know each individual derivative then the result could be computed as the product.
\end{itemize}
\end{frame}

\begin{frame}
    \frametitle{Example}
    Let $f(x)=x^2$, $g(f)=\ln(f)$,$h(g)=g+1$ with $x=3$. We have:
    \begin{align*}
        \frac{df}{dx}&=2x=6\\
        \frac{dg}{df}&=\frac{1}{f}=\frac{1}{9}\\
        \frac{dh}{dg}&=1
    \end{align*}
  Therefore
  \begin{align*}
    \frac{dh}{dx}=1\times\frac{1}{9}\times 6=\frac{2}{3}
  \end{align*}  
An important part of the above is that we know the derivative of \textbf{primitive operations/functions}: even the most complicated of expressions can be broken down into a sequence of primitive operations.
\end{frame}
\begin{frame}
    \frametitle{A second example}
    \begin{itemize}
        \item In the previous example each function depended on a single variable
        \item What happens when a function depends on multiple variables?
        \begin{itemize}
            \item Use partial derivatives
            \item Sum the contributions from different variables
        \end{itemize}
        \item Consider the following computational
    \end{itemize}
   
\begin{align*}
    a&=2\\
    b&=4\\
    c&=a+b\\
    d&=\log(a)*\log(b) \text{  //not a primitive op. More later}\\
    e&=c*d
\end{align*}    

\end{frame}
\begin{frame}
    \frametitle{A second example}
   
\begin{align*}
    c&=a+b\Rightarrow \deriv{c}{a}=1, \deriv{c}{b}=1\\
    d&=\log(a)*\log(b)\Rightarrow \deriv{d}{a}=\frac{\log(b)}{a\ln(2)},\deriv{d}{b}=\frac{\log(a)}{b\ln(2)}\\
    e&=c*d\Rightarrow \deriv{e}{c}=d,\deriv{e}{d}=c
\end{align*}    
\begin{itemize}
    \item Therefore
\end{itemize}
\begin{align*}
    \deriv{e}{a}&=\deriv{e}{c}\deriv{c}{a}+\deriv{e}{d}\deriv{d}{a}=d\times 1+c\times \frac{\log(b)}{a\ln(2)}\\
    \deriv{e}{b}&=\deriv{e}{c}\deriv{c}{b}+\deriv{e}{d}\deriv{d}{b}=d\times 1+c\times \frac{\log(a)}{b\ln(2)}
\end{align*}
\end{frame}

\begin{frame}
    \frametitle{How can the chain rule help with automated differentiation?}
    \begin{itemize}
        \item  Now that we know about the chain rule, how can one use it to compute the gradient?
        \item Construct a graph where each primitive operation is represented by a node
        \item For each such node, attach auxillary nodes to compute its derivative
        \item As an example, let us see how one can automatically computes the following
    \end{itemize}
   
\begin{align*}
    a&=2\\
    b&=4\\
    c&=a+b\\
    d&=\log(a)*\log(b) \text{  //not a primitive op. More later}\\
    e&=c*d
\end{align*}    

\end{frame}
\begin{frame}[t,fragile]{Computational Graph}
 
\begin{columns}[T] % align columns
\begin{column}{.8\textwidth}
%\color{red}\rule{\linewidth}{4pt}

%\tikzset{elliptic state/.style={draw,ellipse}}
\tikzset{
    invisible/.style={opacity=0},
    visible on/.style={alt={#1{}{invisible}}},
    alt/.code args={<#1>#2#3}{%
      \alt<#1>{\pgfkeysalso{#2}}{\pgfkeysalso{#3}} % \pgfkeysalso doesn't change the path
    },
  }
% \tikzstyle{every state}=[
%     draw = black,
%     thick,
%     fill = white,
%     minimum size = 20mm
% ]
\begin{tikzpicture}[
    > = stealth, % arrow head style
    shorten > = 1pt, % don't touch arrow head to node auto,
    node distance = 3cm, % distance between nodes
    semithick, % line style
    transform shape
]


\node[draw=black,ellipse split, visible on=<4->] (e) { 
e=c*d
\nodepart{lower} 
$e=12$
};


\node[draw=black,ellipse split, visible on=<2->] (c) [below left of=e]
{
    c=a+b
    \nodepart{lower}
    c=6
    };

\node[draw=black,ellipse split,visible on=<3->] (d) [below right of=e]
{
    $d=\log a*\log b$
    \nodepart{lower}
    d=2
    };

\node[draw=black,ellipse split] (a) [below left of=c]
{a
\nodepart{lower}
a=2
};

\node[draw=black,ellipse split] (b) [below right of=c]
{b
\nodepart{lower}
b=4
};

\visible<4->{\graph [use existing nodes] {d -> e;};}
%\path[->] (d)  edge (e);
%\path[->]  (c) edge (e);
\visible<4->{\graph [use existing nodes] {c -> e;};}

%\path[->] (a) edge (c);
\visible<2->{\graph [use existing nodes] {a -> c;};}

%\path[->] (b) edge (d);
\visible<3->{\graph [use existing nodes] {b -> d;};}
\visible<3->{\graph [use existing nodes] {a -> d;};}


%\path[->] (d) edge (e);
\visible<2->{\graph [use existing nodes] {b -> c;};}

%\path[->] (b) edge (c);
\end{tikzpicture}
\end{column}%
\hfill%
\begin{column}{.20\textwidth}
\begin{align*}
    \onslide<1->{a&=2\\}
    \onslide<1->{b&=4\\}
    \onslide<2->{c&=a+b\\}
    \onslide<3->{d&=\log(a)*\log(b)\\}
    \onslide<4->{e&=c*d}
\end{align*}
\end{column}%
\end{columns}

\end{frame}
\begin{frame}
    \frametitle{}
    How does the construction of the computational graph help in computing the gradients?
\begin{itemize}
    
    \item While constructing the computational graph, auxillary nodes representing the derivatives are added
    \item Since the graph is a sequence of primitive operations the result of such "local derivatives" can be computed as the graph is constructed  
    \item For example if $z=x+y$, we know that \deriv{z}{x}=1 and \deriv{z}{y}=1
    \item Therefore when node $z$ is created, nodes for \deriv{z}{x}=1 and \deriv{z}{y}=1 are also created
    \item We refer to this phase as "forward pass"
\end{itemize}
\end{frame}

\begin{frame}[t,fragile]
    \frametitle{forward pass}
    \tikzset{
        invisible/.style={opacity=0},
        visible on/.style={alt={#1{}{invisible}}},
        alt/.code args={<#1>#2#3}{%
          \alt<#1>{\pgfkeysalso{#2}}{\pgfkeysalso{#3}} % \pgfkeysalso doesn't change the path
        },
      }
    \begin{tikzpicture}[
        > = stealth, % arrow head style
        shorten > = 1pt, % don't touch arrow head to node
        auto,
        node distance = 2cm, % distance between nodes
        semithick % line style
    ]

    \tikzstyle{every state}=[
        draw = black,
        thick,
        fill = white,
        minimum size = 15mm
    ]
    \node[draw=black,ellipse split,visible on=<4->] (e) {    
    $\scriptstyle e=c*d$
    \nodepart{lower}
    %$\scriptstyle\deriv{e}{e}=1$
    };
    %\node[align=center,anchor=south] (elabel) at (e.south) {e=12};

    \node[rectangle,draw,visible on=<4->] (ec) [below left of=e]{$\scriptstyle\frac{\partial e}{\partial c}=d=2$};
    \node[rectangle,draw,visible on=<4->] (ed) [below right of=e]{$\scriptstyle\frac{\partial e}{\partial d}=c=6$};
    
    \node[draw=black,ellipse split,visible on=<2->] (c) [below left of=ec]{
        $\scriptstyle c=a+b$
        \nodepart{lower}
       
        %{\visible<3->{$\scriptstyle \deriv{e}{c}=2$}}
        };
    %\node[align=center,anchor=south] (clabel) at (c.south) {c=6 };

    \node[draw,ellipse split,visible on=<3->] (d) [below right of=ed]{
        $\scriptstyle d=\log a*\log b$
        \nodepart{lower}
       %{\visible<3->{$\scriptstyle\deriv{e}{d}=6$}}
        };
    %\node[align=center,anchor=south] (dlabel) at (d.south) {d=2};
    
    
    \node[rectangle,draw,visible on=<2->] (ca) [below left of=c]{$\scriptstyle\frac{\partial c}{\partial a}=1$};
    %\node[text width=2cm,visible on=<2->] (commentc) [above left of=ca] {\tiny we know how to compute these};
    \node[rectangle,draw,visible on=<2->] (cb) [below right of=c]{$\scriptstyle\frac{\partial c}{\partial b}=1$};
    
    \node[draw=black,ellipse split,visible on=<1->] (a) [below left of=ca]{
        { $\scriptstyle a \ \ \  $  }
        \nodepart{lower}
        %{\visible<7->{$\scriptstyle\deriv{e}{a}=2+\frac{6}{\ln{2}}$}}
    };
    %\node[align=center,anchor=north] (alabel) at (a.north) {a\\a=2};

    \node[draw=black,ellipse split,visible on=<1->] (b) [below right of=cb]
    {
        $\scriptstyle b\ \ \ $
        \nodepart{lower}
        %{\visible<5->{$\scriptstyle \deriv{e}{b}=2+\frac{6}{4*\ln{2}}$}}
    };
    %\node[align=center,anchor=north] (blabel) at (b.north) {b\\b=4};
    \node[rectangle,draw,visible on=<3->] (db) [below of=d]{$\scriptstyle\frac{\partial d}{\partial b}=\frac{\log a}{b\ln 2}=\frac{1}{4\ln 2}$};

    \node[rectangle,draw,visible on=<3->] (da) [below left of=cb]{$\scriptstyle\frac{\partial d}{\partial a}=\frac{\log b}{a\ln 2}=\frac{1}{\ln 2}$};
    \visible<4->{
        \graph [use existing nodes] {ec -> e;};
        }
    %\path[->] (e) edge (ec) ;
    %\path[->]  (e) edge (ed);
    
    \visible<4->{
        \graph [use existing nodes] {ed -> e;};
        }
    %\path[->]  (ec) edge (c);
    \visible<4->{
        \graph [use existing nodes] {c -> ec;};
        }
    
    \visible<4->{
        \graph [use existing nodes] {d -> ed;};
        }
    %\path[->] (d) edge (ed);
    \visible<2->{
        \graph [use existing nodes] {ca -> c;};
        }
    \visible<2->{
            \graph [use existing nodes] {cb -> c;};
        }
    \visible<3->{
       \graph [use existing nodes] {db -> d;};
       }
    \visible<3->{
        \graph [use existing nodes] {da -> d;};
        }

    \visible<2->{
        \graph [use existing nodes] {a -> ca;};
    }
    \visible<3->{
        \graph [use existing nodes] {b -> db;};
    }
    \visible<2->{
        \graph [use existing nodes] {b -> cb;};
    }
    % \visible<2->{
    %     \graph [use existing nodes] {commentc ->cb;};
    % }
    % \visible<2->{
    %     \graph [use existing nodes] {commentc -> ca;};
    % }
    \visible<3->{
        \graph [use existing nodes] {a -> da;};
    }

\end{tikzpicture}

\end{frame}


\begin{frame}
    \frametitle{Eager vs Graph mode}
\begin{itemize}
    \item Most ML frameworks support two modes of execution
    \begin{enumerate}
        \item Eager mode: operators are executed immediately. This is PyTorch default
        \item Graph mode: the computation starts \textbf{after} the whole  graph for the forward pass is built 
    \end{enumerate}
    \item In eager mode \textbf{no forward pass graph is built}
    \item In our example we built forward pass graphs for purely  illustrative purposes
    \item The graph for computing the gradients (backward pass) \textbf{is built}
    \item Next we describe how the backward pass uses the graph to compute the gradients
    \item The nodes that are actually created in eager mode are \textbf{colored in red}
\end{itemize}
\end{frame}

\begin{frame}[t,fragile]
    \frametitle{Eager mode forward pass}
    \tikzset{
        invisible/.style={opacity=0},
        visible on/.style={alt={#1{}{invisible}}},
        alt/.code args={<#1>#2#3}{%
          \alt<#1>{\pgfkeysalso{#2}}{\pgfkeysalso{#3}} % \pgfkeysalso doesn't change the path
        },
      }
    \begin{tikzpicture}[
        > = stealth, % arrow head style
        shorten > = 1pt, % don't touch arrow head to node
        auto,
        node distance = 2cm, % distance between nodes
        semithick % line style
    ]

    \tikzstyle{every state}=[
        draw = black,
        thick,
        fill = white,
        minimum size = 15mm
    ]
    \node[draw=black,ellipse split,visible on=<4->] (e) {    
    $\scriptstyle e=c*d$
    \nodepart{lower}
    %$\scriptstyle\deriv{e}{e}=1$
    };
    %\node[align=center,anchor=south] (elabel) at (e.south) {e=12};

    \node[rectangle,draw,color=red,visible on=<4->] (ec) [below left of=e]{$\scriptstyle\frac{\partial e}{\partial c}=d=2$};
    \node[rectangle,draw,color=red,visible on=<4->] (ed) [below right of=e]{$\scriptstyle\frac{\partial e}{\partial d}=c=6$};
    
    \node[draw=black,ellipse split,visible on=<2->] (c) [below left of=ec]{
        $\scriptstyle c=a+b$
        \nodepart{lower}
       
        %{\visible<3->{$\scriptstyle \deriv{e}{c}=2$}}
        };
    %\node[align=center,anchor=south] (clabel) at (c.south) {c=6 };

    \node[draw,ellipse split,visible on=<3->] (d) [below right of=ed]{
        $\scriptstyle d=\log a*\log b$
        \nodepart{lower}
       %{\visible<3->{$\scriptstyle\deriv{e}{d}=6$}}
        };
    %\node[align=center,anchor=south] (dlabel) at (d.south) {d=2};
    
    
    \node[rectangle,draw,color=red,visible on=<2->] (ca) [below left of=c]{$\scriptstyle\frac{\partial c}{\partial a}=1$};
    %\node[text width=2cm,visible on=<2->] (commentc) [above left of=ca] {\tiny we know how to compute these};
    \node[rectangle,draw,color=red,visible on=<2->] (cb) [below right of=c]{$\scriptstyle\frac{\partial c}{\partial b}=1$};
    
    \node[draw=black,ellipse split,visible on=<1->] (a) [below left of=ca]{
        { $\scriptstyle a \ \ \  $  }
        \nodepart{lower}
        %{\visible<7->{$\scriptstyle\deriv{e}{a}=2+\frac{6}{\ln{2}}$}}
    };
    %\node[align=center,anchor=north] (alabel) at (a.north) {a\\a=2};

    \node[draw=black,ellipse split,visible on=<1->] (b) [below right of=cb]
    {
        $\scriptstyle b\ \ \ $
        \nodepart{lower}
        %{\visible<5->{$\scriptstyle \deriv{e}{b}=2+\frac{6}{4*\ln{2}}$}}
    };
    %\node[align=center,anchor=north] (blabel) at (b.north) {b\\b=4};
    \node[rectangle,draw,color=red,visible on=<3->] (db) [below of=d]{$\scriptstyle\frac{\partial d}{\partial b}=\frac{\log a}{b\ln 2}=\frac{1}{4\ln 2}$};

    \node[rectangle,draw,color=red,visible on=<3->] (da) [below left of=cb]{$\scriptstyle\frac{\partial d}{\partial a}=\frac{\log b}{a\ln 2}=\frac{1}{\ln 2}$};
    \visible<4->{
        \graph [use existing nodes] {ec -> e;};
        }
    %\path[->] (e) edge (ec) ;
    %\path[->]  (e) edge (ed);
    
    \visible<4->{
        \graph [use existing nodes] {ed -> e;};
        }
    %\path[->]  (ec) edge (c);
    \visible<4->{
        \graph [use existing nodes] {c -> ec;};
        }
    
    \visible<4->{
        \graph [use existing nodes] {d -> ed;};
        }
    %\path[->] (d) edge (ed);
    \visible<2->{
        \graph [use existing nodes] {ca -> c;};
        }
    \visible<2->{
            \graph [use existing nodes] {cb -> c;};
        }
    \visible<3->{
       \graph [use existing nodes] {db -> d;};
       }
    \visible<3->{
        \graph [use existing nodes] {da -> d;};
        }

    \visible<2->{
        \graph [use existing nodes] {a -> ca;};
    }
    \visible<3->{
        \graph [use existing nodes] {b -> db;};
    }
    \visible<2->{
        \graph [use existing nodes] {b -> cb;};
    }
    % \visible<2->{
    %     \graph [use existing nodes] {commentc ->cb;};
    % }
    % \visible<2->{
    %     \graph [use existing nodes] {commentc -> ca;};
    % }
    \visible<3->{
        \graph [use existing nodes] {a -> da;};
    }

\end{tikzpicture}

\end{frame}




\begin{frame}[t,fragile]
    \frametitle{Backward pass}
    \tikzset{
        invisible/.style={opacity=0},
        visible on/.style={alt={#1{}{invisible}}},
        alt/.code args={<#1>#2#3}{%
          \alt<#1>{\pgfkeysalso{#2}}{\pgfkeysalso{#3}} % \pgfkeysalso doesn't change the path
        },
      }
    \begin{tikzpicture}[
        > = stealth, % arrow head style
        shorten > = 1pt, % don't touch arrow head to node
        auto,
        node distance = 2cm, % distance between nodes
        semithick % line style
    ]

    \tikzstyle{every state}=[
        draw = black,
        thick,
        fill = white,
        minimum size = 15mm
    ]

    \node[draw=black,ellipse split] (e) {    
    $\scriptstyle e=c*d$
    \nodepart{lower}
    $\scriptstyle\deriv{e}{e}=1$
    };
    %\node[align=center,anchor=south] (elabel) at (e.south) {e=12};

    \node[rectangle,draw,color=red] (ec) [below left of=e]{$\scriptstyle\frac{\partial e}{\partial c}=d=2$};
    \node[rectangle,draw,color=red] (ed) [below right of=e]{$\scriptstyle\frac{\partial e}{\partial d}=c=6$};
    
    \node[draw=black,ellipse split] (c) [below left of=ec]{
        $\scriptstyle c=a+b$
        \nodepart{lower}
       
        {\visible<3->{$\scriptstyle \deriv{e}{c}=2$}}
        };
    %\node[align=center,anchor=south] (clabel) at (c.south) {c=6 };

    \node[draw,ellipse split] (d) [below right of=ed]{
        $\scriptstyle d=\log a*\log b$
        \nodepart{lower}
        {\visible<3->{$\scriptstyle\deriv{e}{d}=6$}}
        };
    %\node[align=center,anchor=south] (dlabel) at (d.south) {d=2};
    
    
    \node[rectangle,draw,color=red] (ca) [below left of=c]{$\scriptstyle\frac{\partial c}{\partial a}=1$};
    \node[rectangle,draw,color=red] (cb) [below right of=c]{$\scriptstyle\frac{\partial c}{\partial b}=1$};
    
    \node[draw=black,ellipse split] (a) [below left of=ca]{
        $\scriptstyle a$
        \nodepart{lower}
        {\visible<7->{$\scriptstyle\deriv{e}{a}=2+\frac{6}{\ln{2}}$}}
    };
    %\node[align=center,anchor=north] (alabel) at (a.north) {a\\a=2};

    \node[draw=black,ellipse split] (b) [below right of=cb]
    {
        $\scriptstyle b$
        \nodepart{lower}
        {\visible<5->{$\scriptstyle \deriv{e}{b}=2+\frac{6}{4*\ln{2}}$}}
    };
    %\node[align=center,anchor=north] (blabel) at (b.north) {b\\b=4};
    \node[rectangle,draw,color=red] (db) [below of=d]{$\scriptstyle\frac{\partial d}{\partial b}=\frac{\log a}{b\ln 2}=\frac{1}{4\ln 2}$};

    \node[rectangle,draw,color=red] (da) [below left of=cb]{$\scriptstyle\frac{\partial d}{\partial a}=\frac{\log b}{a\ln 2}=\frac{1}{\ln 2}$};
    \visible<2->{
        \graph [use existing nodes] {e -> ec;};
        }
    %\path[->] (e) edge (ec) ;
    %\path[->]  (e) edge (ed);
    
    \visible<2->{
        \graph [use existing nodes] {e -> ed;};
        }
    %\path[->]  (ec) edge (c);
    \visible<3->{
        \graph [use existing nodes] {ec -> c;};
        }
    
    \visible<3->{
        \graph [use existing nodes] {ed -> d;};
        }
    %\path[->] (d) edge (ed);
    \visible<6->{
        \graph [use existing nodes] {c -> ca;};
        }
        \visible<4->{
            \graph [use existing nodes] {c -> cb;};
            }
    \visible<4->{
       \graph [use existing nodes] {d -> db;};
       }
    \visible<6->{
        \graph [use existing nodes] {d -> da;};
        }

       %\path[->] (d) edge (da);
            %\path[->] (d) edge (db);
    %\path[->]  (c) edge (ca);
    %\path[->]  (c) edge (cb);
    %\path[->] (ca) edge (a);
    \visible<7->{
        \graph [use existing nodes] {ca -> a;};
    }
    \visible<5->{
        \graph [use existing nodes] {db -> b;};
    }
    \visible<5->{
        \graph [use existing nodes] {cb -> b;};
    }
    \visible<7->{
        \graph [use existing nodes] {da -> a;};
    }
\end{tikzpicture}
\end{frame}
\begin{frame}
    \frametitle{Details for node d}

    \begin{itemize}
        \item For simplicity and slide space consideration the computation for node $d$ was somehow condensed 
        \item $d=\log(a)*\log(b)$ \textbf{is not a primitive operation}
        \item An ML framework would actually break it down into a sequence of primitive operations by creating temporary, unnamed, nodes
    \end{itemize}
    \begin{align*}
        t_1&=\log(a)\Rightarrow \deriv{t_1}{a}=\frac{1}{a\ln(2)}\\
        t_2&=\log(b)\Rightarrow \deriv{t_2}{b}=\frac{1}{b\ln(2)}\\
        d&=t_1*t_2\Rightarrow \deriv{d}{t_1}=t_2,\deriv{d}{t_2}=t_1
    \end{align*}

\end{frame}

\begin{frame}
    \frametitle{Details for node d}
    \begin{tikzpicture}[
        > = stealth, % arrow head style
        shorten > = 1pt, % don't touch arrow head to node
        auto,
        node distance = 2cm, % distance between nodes
        semithick, % line style
        scale=0.6,
        fontscale/.style=8
    ]

    \tikzstyle{every state}=[
        draw = black,
        thick,
        fill = white,
        minimum size = 10mm
    ]

    \node[ellipse split, draw] (d) {
        $\scriptstyle d=\log(a)*\log(b)=2$
        \nodepart{lower}
         $\scriptstyle\frac{\partial e}{\partial d}=6$
        };
    %\node[align=center,anchor=south] (dlabel) at (d.south) {$\frac{\partial d}{\partial d}=1$};

     \node[rectangle,draw] (dloga) [below left of=d]{$\scriptstyle\frac{\partial d}{\partial \log(a)}=\log(b)=2$};
     \node[rectangle,draw] (dlogb) [below right of=d]{$\scriptstyle\frac{\partial d}{\partial \log(b)}=\log(a)=1$};
        
    \node[draw,ellipse split] (loga) [below left of=dloga]{
        
    $\scriptstyle\log(a)=1$
    \nodepart{lower}
    $\scriptstyle\frac{\partial e}{\partial \log(a)}=12$
    
    };
     %\node[align=center,anchor=south] (logalabel) at (loga.south) {$\frac{\partial d}{\partial \log(a)}=2$};

    \node[ellipse split, draw] (logb) [below left of=dlogb]{
        $\scriptstyle\log(b)=2$
        \nodepart{lower}
        {$\scriptstyle\frac{\partial e}{\partial \log(b)}=6$}
        
        };

    \node[rectangle,draw] (da) [below left of=loga]{$\scriptstyle\frac{\partial \log(a)}{\partial a}=\frac{1}{a*\ln(2)}$};
    \node[rectangle,draw] (db) [below right of=logb]{$\scriptstyle\frac{\partial \log(b)}{\partial b}=\frac{1}{b*\ln(2)}$};
   
    \node[ellipse split,draw] (a) [below left of=da] (a){
        $\scriptstyle a=2$ 
        \nodepart{lower} 
        $\scriptstyle\frac{\partial e}{\partial a}+=\frac{12}{2*\ln(2)}=8.65$
        };
   
    \node[ellipse split, draw] (b) [below left of=db]{
    $\scriptstyle b=4$
    \nodepart{lower}
    $\scriptstyle\frac{\partial d}{\partial b}+=\frac{6}{4*\ln(2)}=2.16$
    };

   
     \path[->] (d) edge (dloga) ;
    \path[->] (d)  edge (dlogb);
     \path[->]  (dloga) edge (loga);
     \path[->]  (loga) edge (da);
     \path[->]  (da) edge (a);


     \path[->]  (dlogb) edge (logb);
     \path[->]  (logb) edge (db);
     \path[->]  (db) edge (b);
   
\end{tikzpicture}
\end{frame}
\begin{frame}
    \frametitle{Example}
    to compute $d=a+b+c$? in graph mode, 
\begin{enumerate}
    \item how many nodes are needed (without gradient nodes) 
    \item Including the gradient nodes?

\end{enumerate}
    

\end{frame}


\begin{frame}
    \frametitle{Graph for $d=a+b+c$}
    \begin{tikzpicture}[
        > = stealth, % arrow head style
        shorten > = 1pt, % don't touch arrow head to node
        auto,
        node distance = 2cm, % distance between nodes
        semithick, % line style
        scale=0.6,
        fontscale/.style=8
    ]

    \tikzstyle{every state}=[
        draw = black,
        thick,
        fill = white,
        minimum size = 10mm
    ]

    \node[circle,draw] (d) {
        $\scriptstyle d=t+c$
        };

     \node[rectangle,draw] (dddt) [below left of=d]{$\scriptstyle\frac{\partial d}{\partial t}=1$};
     \node[rectangle,draw] (dddc) [below right of=d]{$\scriptstyle\frac{\partial d}{\partial c}=1$};
        
    \node[circle,draw] (t) [below left of=dddt]{
    $\scriptstyle  t=a+b$
    };
    \node[circle,draw,minimum size=10mm] (c) [below right of=dddc]{
    $\scriptstyle  c$
    };
     %\node[align=center,anchor=south] (logalabel) at (loga.south) {$\frac{\partial d}{\partial \log(a)}=2$};

    \node[draw] (dtda) [below left of=t]{
        $\scriptstyle\deriv{t}{a}=1$
        };
    \node[draw] (dtdb) [below right of=t]{
            $\scriptstyle\deriv{t}{b}=1$
    };
            
    \node[circle,draw,minimum size=10mm] (a) [below left of=dtda]{
            $\scriptstyle  a$
    };     
    
    \node[circle,draw,minimum size=10mm] (b) [below right of=dtdb]{
            $\scriptstyle  b$
    };     
       
    \path[->] (a) edge (dtda) ;
    \path[->] (b)  edge (dtdb);
    \path[->]  (dtda) edge (t);
      \path[->]  (dtdb) edge (t);
    \path[->]  (t) edge (dddt);
    \path[->]  (dddt) edge (d);
    \path[->]  (c) edge (dddc);
    \path[->]  (dddc) edge (d);
   
\end{tikzpicture}
\end{frame}



\begin{frame}
    \frametitle{PyTorch Details}
\begin{itemize}
    \item PyTorch implements the above using a graph of gradient functions: \lstinline{grad_fn}
    \item The values in rectangular nodes represent the outputs of those functions
    \item  The starting point is \lstinline{e.grad_fn} with input 1. The output of \lstinline{e.grad_fn(1.)} is \lstinline{[2,6]} as expected.
    \item The output \lstinline{[2,6]} is used as input to the functions \lstinline{e.grad_fn.next_functions}
    \item The code for the whole graph is shown on the corresponding notebook
\end{itemize}
\end{frame}
\end{document}
